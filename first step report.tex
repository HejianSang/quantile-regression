\documentclass[]{article}
\usepackage{booktabs}
\usepackage{multirow}
\usepackage{amssymb}
%\usepackage{setspace}
\usepackage{latexsym}
\usepackage{amsmath}
\usepackage{graphicx}
\usepackage{caption}
\usepackage{float}
\usepackage{subfigure}
\usepackage{bm}
\usepackage{color}
\usepackage[utf8]{inputenc}
\usepackage[english]{babel}
\newtheorem{theorem}{Theorem}

%opening
\title{Draft: Quantile estimation with using auxiliary information}


\begin{document}

\maketitle

\section{Motivation}
Suppose we have a population, we can always observe the information X. But we can only observe the sample of $Y$ from the population. Our target is to estimate the population quantile of $Y$.

\section{Method}
\subsection{Direct Method}
A very naive method is to use the sample of $Y$. Define $$F_n(y)=\frac{\sum_{i\in A} d_i I\left(y_i\leq y \right) }{\sum_{i\in A}d_i}$$. 
Then we can get the estimated quantile by $$\hat \theta_d=\inf\left\lbrace y:F_n(y)\geq \tau \right\rbrace $$

\subsection{Regression estimator and difference estimate under pseudo empirical log-likelihood function}

We put these two estimators together, because they are asymptotically same. The theorem 1 in the paper shows that. $$\hat \theta_{df}=\hat{\theta_d}+N^{-1}\left\lbrace \sum_{i=1}^{N} q\left(\textbf x_i;\hat\beta_{\tau_0}\right) - \sum_{i\in A} d_i q\left(\textbf x_i;\hat\beta_{\tau_0} \right) \right\rbrace $$

For this estimator, we can extend to multi-calibrations. We can use $\tau_0,\tau_1,...,\tau_m$ to do the calibration together.

\subsection{Using empirical distribution to do calibration}
Define $$F_N(x)=\frac{\sum_{i}^{N}I(x_i \leq x)}{N}$$
$$F_w(x)=\sum_{i\in A} w_i I(x_i \leq x) $$

We want to minimize
\begin{eqnarray}
&& \arg\min_{w} Var\left( \hat\theta_w\right) \approxeq \left(\frac{1}{f(\hat\theta_{\tau})} \right)^2 Var \left(\hat F_w(\theta)\right) \\
&& \text{Given constraints:}\nonumber\\
&& F_N(x)=F_w(x)\nonumber\\
&& \sum_{i\in A} w_i=1\nonumber
\end{eqnarray}
But unfortunately, this constraint is not always holds for any $x\in \mathbb{R}$. So I just use the moments to do the calibration. Because the theorem below:
\begin{theorem}
	(Frechet-Shohat Therom) Suppose ${X_n},n \geq 0$ are random variables. If $\varliminf EX_n^r=\beta_r$ for all $r$ and if all $\beta_r$ are the moments of a unique random variable $X_0$, then $X_n\longrightarrow X_0$ in distribution.
\end{theorem}

So we can use the first four moments to do the calibration. But I don't get the minimized weights.  I first to do the simulation to see if this works. Then we can minimize the $w$ to see we can get the improvement.

\subsection{Ratio estimator}
For any quantile $\tau$, we can get the true population quantile for $X$. So the ratio estimator is possible. Suppose the $\theta_{x,\tau}$ is the quantile of $\tau$ for population $X$.

The ratio estimator is $$ \hat\theta_{ratio}= \theta_{x,\tau}\frac{\hat\theta_{w,y}}{\hat\theta_{w,x}} $$
 But in some cases if the $\hat\theta_{w,x}$ is 0 or closed to 0. That may cause problems.
 We may use the bias-corrected ratio estimator. 
 $$ \hat\theta_{ratio}= \theta_{x,\tau}\frac{\hat\theta_{w,y}\hat\theta_{w,x}+\hat C\left(\hat\theta_{w,y},\hat\theta_{w,x} \right) }{\hat\theta_{w,x}^2+\hat V\left( \hat\theta_{w,x}\right) } $$
 
 The problem is that can we use $$\hat C\left(\hat\theta_{w,y},\hat\theta_{w,x} \right) =??$$
 and $$ \hat  V\left( \hat\theta_{w,x}\right) \approxeq \left(\frac{1}{f(\hat\theta_{\tau})} \right)^2 \hat Var \left(\hat F_w(\theta)\right)$$
 
 \section{A small simulation study}
 The simulation studies are conducted to compare the performance of different estimators. Two finite populations of size $N=1000$ were generated from bivariate normal distribution respectively. The correlation between two variables in the first population is 0.9, then 0.6. For each simulation run , a simple random sample of size $n=100$ was taken.
 
 To compare with each other, we set the direct estimator as the base and define relative efficiency: $$RE_{*}=\frac{MSE_{*}}{MSE_{Direct}}$$.
 
 Note: We use the same $\tau_0=0.5$ to get the difference estimator.
 
 
 
\end{document}
